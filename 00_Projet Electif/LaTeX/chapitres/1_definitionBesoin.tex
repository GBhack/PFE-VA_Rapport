Une définition pertinente du besoin est une condition absolument nécessaire pour la bonne réalisation de tout projet. Nous nous emploierons donc à définir aussi précisément et pertinemment que possible le besoin motivant ce projet, et à régulièrement revenir sur ce dernier afin de prendre en compte d'éventuelles évolutions et à prévenir toute dérive du projet.

\subsection{Besoin}
Le besoin à l'origine de ce projet fut exprimé par mesdames A. Belbachir et S. Benabid en tant qu'enseignants chercheurs au laboratoire Mécatronique, Signal et Systèmes à l'IPSA. Leur souhait était de bénéficier d'une plateforme articulée autour de robots autonomes évoluant dans une simulation d’environnement urbain, capables de détecter des feux de signalisation "intelligents" et d'adapter leur comportement en conséquence. Cette plateforme pourrait servir de support de TP\nomenclature{TP}{Travaux Pratiques} dans l'ensemble des matières enseignées par le département, mais également servir de "vitrine" voir même de plateforme de recherche.\\

Les principales fonctionnalités évoquées étant :
\begin{itemize}
	\item \textbf{La reconnaissance d'image} pour la détection des feux de signalisation.
	\item \textbf{La présence de feux bicolores commandés par FPGA\nomenclature{FPGA}{PGA (Field-Programmable Gate Array, ou Réseau de Portes Programmables : un type très répandu de circuit logiques programmables.}.} Ces feux seraient "intelligents" dans le sens où il s'adapteraient à la circulation. La présence de capteurs de présence est donc induite.
\end{itemize}

\subsection{Précision du besoin}

Ayant suivi nombre de matières enseignées par le département et participé à de nombreuses séances de travaux pratiques, nous bénéficions d'une image relativement claire des implications de ce projet ainsi que des contraintes relevant pour la plupart du simple bon sens.
Dans un soucis de clarté et d'application de "bonnes pratiques" nous prîmes cependant soin de valider l'ensemble de ces éléments au cours de réunions avec nos "clients" du laboratoire.\\

Ainsi, nous ajoutâmes les points suivants à la liste des exigences :
\begin{itemize}
	\item "Côté circuit" :
	\begin{itemize}
		\item \textbf{Le circuit devra bénéficier d'un encombrement raisonnable}.
		\item \textbf{Les feux de circulation devront pouvoir être commandés via une carte FPGA}.
	\end{itemize}
	\item "Côté robots" :
	\begin{itemize}
		\item \textbf{Les robots devront pouvoir être utilisés en classe sans que les préoccupations matérielles ne soient accaparantes}.
		\item \textbf{Les robots devront pouvoir être programmés en utilisant les langages enseignés à l'IPSA} à savoir C et C++ et ses variantes (Arduino...), Python, voir même Matlab...
		\item \textbf{Les robots devront bénéficier de possibilités d'application flexibles :} les enseignements étant ciblés, il est important que les utilisateurs des robots puissent se concentrer sur un aspect de leur utilisation sans avoir à se soucier des autres. De même, les robots devront embarquer suffisamment de capteurs ou tout du moins de capacité d'extensions pour que cela ne soit pas un facteur limitant lors de l'élaboration de sujets de TP.
		\item \textbf{Les robots devront ne pas pouvoir représenter un danger pour ses utilisateurs}.
	\end{itemize}
\end{itemize}