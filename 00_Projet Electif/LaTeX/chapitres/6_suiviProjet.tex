Un suivi de projet rigoureux devra être mis en place.\\

Nous recommandons l'instauration d'un compte-rendu hebdomadaire (qui pourra être fait par mail), ainsi que d'une réunion d'avancement mensuelle.\\

Il serait également judicieux de se contraindre à rédiger quotidiennement un journal de travail.\\

Surtout, un tableau partagé permettra de suivre l'avancement "en temps réel" et, si besoin, de réajuster les prévisions et l'allocation des ressources.\\

Nous avons produit le "code de bonne pratiques" suivant afin d'avoir une référence à laquelle se rapporter au cours du projet:\\
\begin{adjustwidth}{1cm}{0.5cm}
	\fbox{\parbox{\linewidth-2\fboxrule-2\fboxsep}{
	Ce guide de bonnes pratiques est à consulter régulièrement et respecter tout au long du projet.\\
	Ce guide a vocation à améliorer la productivité et non l'inverse : les protocoles "lourds" et présentant peu d'intérêt sont à proscrire.\\

	\begin{enumerate}
		\item \textbf{COMMUNICATION :}
		\begin{itemize}
			\item \textbf{EMAILS :}\\
			Pour informations qui doivent rester, et/ou échange avec différents acteurs\\
			\textbf{Objet :} [PFE VA] - [CATEGORIE] Objet	\textit{(ex: [PFE VA] - [FINANCIER] Dépassement de budget)}\\
			\textbf{Contenu :} Concis, en "points par points"\\
			Utiliser les champs "Destinataire, "Copie Carbone" et "Copie Carbone Invisible" à bon escient (Destinataire pour personne directement visée, CC pour acteurs à tenir informés)\\
			Essayer de répondre en point par point sur la structure du mail initial\\

			\item \textbf{SLACK :}\\
			Pour communication "temps réel", conversation. Pas de décisions importantes.\\
			Diviser en canaux par thématique\\
		\end{itemize}

		\item \textbf{STOCKAGE :}\\
		Respecter l'architecture en place\\
		Noms de fichiers doivent être clairs, explicites et concis\\
		Sauvegardes régulières sur support externe\\
	\end{enumerate}}}
\end{adjustwidth}
\begin{adjustwidth}{1cm}{0.5cm}
	\fbox{\parbox{\linewidth-2\fboxrule-2\fboxsep}{
	\begin{enumerate}
	\setcounter{enumi}{2}
		\item \textbf{GIT :}\\
		Toujours commencer session de travail par un pull\\
		Un commit à chaque "bloc de modification" : ajout d'une fonction, correction d'un bug... Pas de commit sur une modif non testée\\
		Création d'une branche pour chaque modification "expérimentale". Fusion de la branche si concluante, après concertation.\\
		Exclure des commits tous les fichiers binaires (sauf illustrations quand utilisées par le code) et tous les fichiers générés (fichiers de compilation...) : git ne doit concerner que les fichiers textes (sauf fichiers nécessaires à la bonne compilation/utilisation du code). => utiliser le .gitignore\\
		=> un pull doit toujours être suivi d'une recompilation : ne jamais  utiliser un fichier déjà compilé (qui ne doivent de toute façon pas être commit).

		
		\item \textbf{CODE :}
		Respect de la convention décrite ici : \url{http://all4dev.libre-entreprise.org/index.php/Conventions_de_syntaxe_en_python}
		Respect de la convention PEP8
		Ajout systématique d'un docstring avec description de la fonction (ou classe), des paramètres et du retour.

		\item \textbf{SUIVI :}
		Utilisation de ProjectLibre pour ajuster le planning et l'avancement en temps réel
		Un point d'avancement par email 1 fois par semaine avec accomplissements, problèmes rencontrés, besoins éventuels et modifications notables du projet.

		\item \textbf{DOCUMENTATION :}
		\begin{itemize}
			\item \textbf{EXISTANTE :}\\
			Prendre soin de conserver toute la documentation dans le répertoire prévu à cet effet. Veiller à suivre une nomenclature et une organisation simple et explicite.\\
			Compléter la bibliographie au fur et à mesure via le formulaire dédié.\\
			\item \textbf{PRODUITE :}\\
			Au fur et à mesure du développement, utiliser les commentaires et les docstrings pour documenter les programmes.\\
			Enregistrer dans de simples fichiers textes, ou communiquer par email s'ils doivent être partagés les points à prendre en compte \textit{(ex: Attention à toujours couper le contact avant de charger un programme !)}\\
			La construction de la documentation livrable ne débutera qu'en fin de projet, quand nous bénéficierons d'une vue d'ensemble.\\
		\end{itemize}
	\end{enumerate}}}
\end{adjustwidth}