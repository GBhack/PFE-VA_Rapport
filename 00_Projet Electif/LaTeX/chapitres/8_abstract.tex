Ce dossier constitue l'\textbf{étude préliminaire} d'un projet de réalisation de \textbf{plateforme d'enseignement et de recherche} pour le pôle Mécatronique, Signal et Systèmes de la Direction de la Recherche et de l’Innovation de l’IPSA (DRII) de l'IPSA, école d'ingénieur de l'air et de l'espace.\\

Cette plateforme, composée d'un \textbf{"circuit" interactif} modelisant un réseau routier urbain et de \textbf{"robots" autonomes} capables de se déplacer en autonomie et en respect des règles de la route sur ce circuit, pourra servir de \textbf{support de travaux pratiques} pour l'ensemble des enseignements du pôle, voir même de \textbf{support de recherche}.\\

Le présent dossier propose une \textbf{analyse détaillée} de ce \textbf{besoin} et de ses implications, ainsi que la description d'une \textbf{solution technique} complète et argumentée. Enfin, une estimation des \textbf{coûts} et des \textbf{plannings} est offerte.\\

Ce dossier est agrémenté de diverses illustrations, schémas, plans et vues 3D, formant ainsi un guide presque suffisant pour la réalisation du projet.